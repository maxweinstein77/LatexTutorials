\documentclass[11pt]{article}
\usepackage{amsfonts, amssymb, amsmath}
\usepackage{float}
\parindent 0px % Prevent indentations
\pagestyle{empty} % Turns off page numbering

\begin{document}

The distributive property states that $a(b+c)=ab+ac$, for all $a, b, c \in \mathbb{R}$. \\[6pt]
The equivalence class of $a$ is $[a]$. \\[6pt]
The set $A$ is defined to be $\{1, 2, 3\}$. \\[6 pt] % Have to put slash in front of brackets for them to appear.
The movie tickets costs $\$11.50$. % Again, have to put slash in front of $ for it to show up.

$$2\left(\frac{1}{x^2-1}\right)$$ % This is how to make parentheses properly enclose full height of fraction.
$$2\left[\frac{1}{x^2-1}\right]$$
$$2\left\{\frac{1}{x^2-1}\right\}$$ % Remember to put slash in front of {} too.
$$2\left \langle \frac{1}{x^2-1}\right \rangle$$ % Angular brackets
$$2\left | \frac{1}{x^2-1}\right | $$ 
$$\left.\frac{dy}{dx}\right|_{x=1}$$ % Put period after \left to prevent it from being run.
$$\left(\frac{1}{1+\left(\frac{1}{1+x}\right)}\right)$$

Tables:\\

\begin{tabular}{|c||c|c|c|c|c|} % Center aligned
\hline
$x$ & 1 & 2 & 3 & 4 & 5 \\ \hline
$f(x)$ & 10 & 11 & 12 & 13 & 14 \\ \hline
\end{tabular}

\vspace{1cm}

\begin{table}[H]
\centering
\def\arraystretch{1.5}
\begin{tabular}{|c||c|c|c|c|c|}
\hline
$x$ & 1 & 2 & 3 & 4 & 5 \\ \hline
$f(x)$ & $\frac{1}{2}$ & 11 & 12 & 13 & 14 \\ \hline
\end{tabular}
\caption{These values represent the function $f(x)$.}
\end{table}


\begin{table}[H]
\centering
\caption{The relationship between $f$ and $f'$.}
\def\arraystretch{1.5}
\begin{tabular}{|l|p{3in}|}
\hline
$f(x)$ & $f'(x)$ \\ \hline
$x>0$ & The function $f(x)$ is increasing. The 
function $f(x)$ is increasing. The function $f(x)$ is 
increasing. The function $f(x)$ is increasing. The 
function $f(x)$ is increasing. \\ \hline
\end{tabular}
\end{table}

Arrays:
\begin{align}
5x^2-9=x+3\\
5x^2-x-12=0 % Force space with \, or do \text for text in align
\end{align}

\begin{align*} % Put * after align to get rid of numbers
5x^2-9&=x+3\\ % Use & before = to make equal signs line up
5x^2-x-12&=0\\
&=12+x-5x^2 % Use & before = to make equal signs line up
\end{align*} % Put * after align to get rid of numbers

\end{document}