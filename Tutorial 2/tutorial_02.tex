\documentclass[11pt]{article}
\pagestyle{empty}
\usepackage{amsmath, amssymb, amsfonts}

\begin{document}
superscripts $$2x^3$$
$$2x^{34}$$
$$2x^{3x+4}$$
$$2x^{3x^4+5}$$

subscripts
$$x_1$$
$$x_{12}$$
$$x_{1_2}$$
$$x_{1_{2_3}}$$
$$a_0, a_1, a_2, \ldots, a_{100}$$

Greek letters
$$\pi$$
$$\Pi$$
$$\alpha$$
$$A=\pi r^2$$

Trig functions
$$y=\sin x$$
$$y=\cos x$$
$$y=\csc \theta$$
$$y=\sin ^{-1} x$$
$$y=\arcsin x$$

Log functions
$$y=\log x$$
$$y=\log_5 x$$
$$y=\ln x$$

Roots
$$\sqrt{2}$$
$$\sqrt[3]{2}$$
$$\sqrt{x^2+y^2}$$
$$\sqrt{1+\sqrt{x}}$$

Fractions
$$\frac{2}{3}$$
About $\displaystyle \frac{2}{3}$ of the glass is full.\\[16pt]
About $\frac{2}{3}$ of the glass is full.\\[6pt]
About $\dfrac{2}{3}$ of the glass is full.

$$\frac{\sqrt{x+1}}{\sqrt{x+2}}$$
$$\frac{\sqrt{x+1}}{\sqrt{x+2}}$$

$$\frac{1}{     1+\frac{1}{x}      }$$

\end{document}

\begin{comment}
    When making exponents with more than one digit,
    must use curly brackets to ensure entire value
    is placed in exponent notation.

    Same thing applies to subscripts. 

    Double subscript: $$x_{1_{2_3}}$$
    Tripe subscript: $$x_{1_{2_{3_4}}}$$

    Sequence: $$a_0, a_1, a_2, \ldots, a_{100}$$

    $$A=\pi r^2$$  MUST HAVE SPACE BETWEEN pi and r,
    because pir is not a command!

    Interprets s i and n as variables: $$y=sin x$$
    Interprets sin and the trig function: $$y=\sin x$$

    Normal math sentence: About $\frac{2}{3}$ of the glass is full.
    Math sentence but with equation part enlarged: About $\displaystyle \frac{2}{3}$ of the glass is full.
    Or with packages you can do: About $\dfrac{2}{3}$ of the glass is full.

    To insert vertical space between two lines, add \\[6pt] at previous line.
    You can do normal way by adding \\ at previous line then leaving a line of space 
    between two lines of code but the problem is that it will indent the next paragraph.

    Line breaks: add \\[16pt] at previous line.
    \end{comment}